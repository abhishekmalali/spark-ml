\documentclass{article}

\usepackage{amsmath}
\usepackage{graphicx}
\usepackage{courier}
\usepackage{multicol}
\usepackage{microtype} %required to fix courier linebreaking issues
\usepackage{hyperref} % make references into links
\usepackage{amsfonts}
\hypersetup{colorlinks=true, linkcolor=blue} %make them pretty links
\usepackage[all]{hypcap} %make them link to the right place
\usepackage{url}
\makeatletter
\g@addto@macro{\UrlBreaks}{\UrlOrds}
\makeatother
\usepackage{geometry}
\geometry{legalpaper, portrait, margin=1in}

\begin{document}
	\title{Machine Learning Algorithms in Spark}
	\author{Abhishek Malali (abhishekmalali@g.harvard.edu)\\
			Neil Chainani (chainani@g.harvard.edu)\\
			Leonhard Spiegelberg (spiegelberg@g.harvard.edu)}
	\date{\today}
	\maketitle
	\section{Background}
	Spark has a machine learning library called MLib which has a missing Neural networks functionality which serves as our main inspiration to implement additional ML algorithms in Spark.

	\section{Benchmarking Data}
	We intend to use the (small dataset) for initial testing which we would later scale up to (large dataset) to benchmark spark implementations against scikit-learn. 


	\section{Objectives - Functionality and Performance}

	\section{Algorithms to be implemented}
		\begin{itemize}
			\item Artificial Neural Networks - ANNs are highly paralellizable and plan to implement a basic implementation which works on Backpropagation for starters which can be extended to other learning algorithms.
			\item Random Forests - Random forests are paralellizable since we need to create different classification trees which can be done on multiple processors without conflict and later integrated to find results.
			\item Add one more ML algorithm
		\end{itemize}

	\section{Design Overview - Technologies and use of Parallelism}

	\section{Verification}

	\section{Schedule, Milestones, and Division of Work}
\end{document}