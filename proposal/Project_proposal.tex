\documentclass{article}

\usepackage{amsmath}
\usepackage{graphicx}
\usepackage{courier}
\usepackage{multicol}
\usepackage{microtype} %required to fix courier linebreaking issues
\usepackage{hyperref} % make references into links
\usepackage{amsfonts}
\hypersetup{colorlinks=true, linkcolor=blue} %make them pretty links
\usepackage[all]{hypcap} %make them link to the right place
\usepackage{url}
\makeatletter
\g@addto@macro{\UrlBreaks}{\UrlOrds}
\makeatother
\usepackage{geometry}
\geometry{legalpaper, portrait, margin=1in}

\begin{document}
	\title{Machine Learning Algorithms in Spark}
	\author{Abhishek Malali (abhishekmalali@g.harvard.edu)\\
			Neil Chainani (chainani@g.harvard.edu)\\
			Leonhard Spiegelberg (spiegelberg@g.harvard.edu)}
	\date{\today}
	\maketitle
	\section{Background}
	
	Spark's machine learning library MLlib has vast functionality but unfortunately lacks certain classifiers. Ordinal logistic regression is one of those missing classifiers, and is particularly useful for ordinal dependent variables, such as rating systems and surveys. We propose to design an ordinal logit library built on top of Spark that is optimized to take advantage of Spark's parallelism.  

	\section{Benchmarking Data}
	We intend to use the (small dataset) for initial testing which we would later scale up to (large dataset) to benchmark spark implementations against scikit-learn. 


	\section{Objectives - Functionality and Performance}

	\section{Algorithms to be implemented}
		\begin{itemize}
			\item Artificial Neural Networks - ANNs are highly paralellizable and plan to implement a basic implementation which works on Backpropagation for starters which can be extended to other learning algorithms.
			\item Random Forests - Random forests are paralellizable since we need to create different classification trees which can be done on multiple processors without conflict and later integrated to find results.
			\item Add one more ML algorithm
		\end{itemize}

	\section{Design Overview - Technologies and use of Parallelism}

	\section{Verification}
	Once a data set is determined, we will use the existing ordinal logit libraries in Mort or scikit-learn to establish a baseline. We will evaluate performance as compared against other Spark implementations of ordinal logit found online (if they exist) and against MLlib's linear regression, by treating the ordinal data as continuous. 

	\section{Schedule, Milestones, and Division of Work}
\end{document}