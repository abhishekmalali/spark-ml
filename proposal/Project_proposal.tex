\documentclass{article}

\usepackage{amsmath}
\usepackage{graphicx}
\usepackage{courier}
\usepackage{multicol}
\usepackage{microtype} %required to fix courier linebreaking issues
\usepackage{hyperref} % make references into links
\usepackage{amsfonts}
\hypersetup{colorlinks=true, linkcolor=blue} %make them pretty links
\usepackage[all]{hypcap} %make them link to the right place
\usepackage{url}
\makeatletter
\g@addto@macro{\UrlBreaks}{\UrlOrds}
\makeatother
\usepackage{geometry}
\geometry{legalpaper, portrait, margin=1in}

\begin{document}
	\title{Machine Learning Algorithms in Spark}
	\author{Abhishek Malali (abhishekmalali@g.harvard.edu)\\
			Neil Chainani (chainani@g.harvard.edu)\\
			Leonhard Spiegelberg (spiegelberg@g.harvard.edu)}
	\date{\today}
	\maketitle
	\section{Background}
	Spark has a machine learning library called MLib which has a missing Neural networks functionality which serves as our main inspiration to implement additional ML algorithms in Spark.

	
	\section{Objectives - Functionality and Performance}
	To implement the machine learning algorithms in Spark while benchmarking the code against Scikit-learn and MLlib which is the standard Machine learning library for Spark. A part of the project will also focus on fitting the data to the problem in the best possible way while ensuring all standard ML rules are followed.


	\section{Milestones}
		\begin{itemize}
			\item 11/13 - Research and benchmark implementations(mord/scikit-learn)
			\item 11/20 - Preliminary implementations in Spark for ordinal regressions
			\item 11/27 - Parallelizing regression code and work on bayesian ordinal regressions
			\item 12/4 - Final analysis of current work and start work on project webpage
		\end{itemize}
	
	\section{Division of work}
		\begin{itemize}
			\item Leonhard - Design of algorithms for Ordinal regressions
			\item Neil - Implementation of benchmark code and implementation of algorithms in Spark
			\item Abhishek - Implementing algorithms in Spark and Parallelizing strategies
		
\end{document}